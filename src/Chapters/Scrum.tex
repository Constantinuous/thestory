\subsection{Scrum}
Scrum is a framework for developing and sustaining complex products by Jeff Sutherland and Ken Schwaber. Scrum certification is provided by Scrum Alliance (Chairman Jeff Sutherland \url{https://www.scrumalliance.org/}) and Scrum.org (Founded by Ken Schwaber \url{https://www.scrum.org/}) which together update the official Scrum Guide\cite{scrum:guide}.\newline

Most of what I summarize here can be found directly in the Scrum Guide. Please remember that the Scrum Guide also improves over the years. Those that are still used to Sprint Planning 1 and 2 will find that these meeting no longer exist in that form. Remember, even though Scrum Certification is split between two different companies, they are both working on the same Scrum Guide.\newline

Scrum is free and offered in the Scrum Guide\cite{scrum:guide}. Scrum’s roles, artifacts, events, and rules are immutable and although implementing only parts of Scrum is possible, the result is not Scrum. Scrum exists only in its entirety and functions well as a container for other techniques, methodologies, and practices.

\subsubsection{Overview}

Scrum is founded on empirical process control theory, or empiricism. Empiricism asserts that knowledge comes from experience and making decisions based on what is known. Scrum employs an iterative, incremental approach to optimize predictability and control risk. Three pillars uphold every implementation of empirical process control: transparency, inspection, and adaptation.

Scrum knows \textsc{3 roles}, \textsc{3 artifacts} and \textsc{5 events}.\newline

\fbox{\begin{minipage}{30em}
	Scrum \textbf{does not know} User Stories. Scrum only deals with Product Backlog Items (PBI). It is however common to use User Stories for PBIs.
\end{minipage}}

\subsubsection{3 Roles}
\begin{figure}[H] % Example image
\center{\includegraphics[width=1\linewidth]{Scrum/Roles}}
\caption{Scrum Roles}
\label{fig:scrum:roles}
\end{figure}

\paragraph{Development Team}
\begin{description}
   \item [Name] Dev Team
\end{description}
The Development Team should have a T \cite{wiki:tshaped} or $\pi$ shape. That means that the team should consist of people with a generalist (broad) skill set who are specialists in one (T-shape) specific area. Eventually that team can mature to have generlists who are specialists in two ($\pi$-shape) areas.

\subsubsection{3 Artifacts}

The \textbf{Product Backlog} is an ordered list of everything that might be needed in the product and is the single source of requirements for any changes to be made to the product. The Product Owner is responsible for the Product Backlog, including its content, availability, and ordering. Requirements never stop changing, so a Product Backlog is a living artifact. Multiple Scrum Teams often work together on the same product. One Product Backlog is used to describe the upcoming work on the product.

Product Backlog refinement is the act of adding detail, estimates, and order to items in the Product Backlog. This is an ongoing process in which the Product Owner and the Development Team collaborate on the details of Product Backlog items. Product Backlog items can be updated at any time by the Product Owner or at the Product Owner’s discretion. roduct Backlog items that will occupy the Development Team for the upcoming Sprint are refined so that any one item can reasonably be “Done” within the Sprint time-box. Product Backlog items that can be “Done” by the Development Team within one Sprint are deemed “Ready” for selection in a Sprint Planning.\newline

The \textbf{Sprint Backlog} is the set of Product Backlog items selected for the Sprint, plus a plan for delivering the product Increment and realizing the Sprint Goal. The Sprint Backlog is a forecast by the Development Team about what functionality will be in the next Increment and the work needed to deliver that functionality into a “Done” Increment. Only the Development Team can change its Sprint Backlog during a Sprint. The Sprint Backlog is a highly visible, real-time picture of the work that the Development Team plans to accomplish during the Sprint, and it belongs solely to the Development Team. At any point in time in a Sprint, the total work remaining in the Sprint Backlog can be summed to monitor the progress.\newline

The \textbf{Increment} is the sum of all the Product Backlog items completed during a Sprint and the value of the increments of all previous Sprints. At the end of a Sprint, the new Increment must be “Done,” which means it must be in useable condition and meet the Scrum Team’s definition of “Done.” It must be in useable condition regardless of whether the Product Owner decides to actually release it.

\subsubsection{5 Events}

\fbox{\begin{minipage}{35em}
\begin{description}
   \item [Name] Daily Stand-Up
   \item [Duration] Max. 15m
   \item [Participants] Scrum Team (PO optional), open for everyone
   \item [Responsible] Scrum Master
   \item [Input] ?
   \item [Meeting Goal] Shared understanding of where they are
   \item [Questions] \hfill \\ 
    What did I do yesterday that helped the Development Team meet the Sprint Goal?\hfill \\ 
    What will I do today to help the Development Team meet the Sprint Goal?\hfill \\ 
    Do I see any impediment that prevents me or the Development Team from meeting the Sprint Goal?

\end{description}
\end{minipage}}
\newline\newline Only the committed Scrum Team is allowed to talk. Other participants can request talking rights from the Scrum Master. The Daily is almost self moderating. The \textbf{Rule of two Hands} allows two or more independent people to raise their hands when a topic for them is unimportant. Then the topic is dropped immediately. You can also ask the team if they believe that they can fulfill the sprint goal. The Scrum Team Members vote with their thumbs (Up, middle, down) which leads to more discussions. You can also update the Sprint Burndown Chart.\newline

\fbox{\begin{minipage}{35em}
\begin{description}
   \item [Name] Sprint Planning
   \item [Duration] Max. 8h
   \item [Participants] Scrum Team, Stakeholder
   \item [Responsible] Scrum Master
   \item [Input] Ordered, sufficiently filled Backlog\hfill \\ 
		A Sprint Goal by the PO (or the Dev Team)
   \item [Meeting Goal] Shared understanding of what to do and how
   \item [Output] Sprint Backlog (Commitment of the Scrum Team)
\end{description}
\end{minipage}}
\newline\newline Text about the Sprint Planning\newline

\fbox{\begin{minipage}{35em}
\begin{description}
   \item [Name] Sprint Review
   \item [Duration] Max. 4h (30m to 1h is sensible)
   \item [Participants] Scrum Team, Key Stakeholders (invited by PO)
   \item [Responsible] Scrum Master
   \item [Input] Potentially Shippable Product (in neutral customer environment)\hfill \\ 
		Product Owner ensures Product Backlog and Roadmap are presentable.
   \item [Meeting Goal] Know where we are and how to continue.
   \item [Output] Feedback, Customer Happiness, Revised Product Backlog that defines the probable Product Backlog items for the next Sprint
	 \item [Note] During the Review the PO can actively manage the present Stakeholders. Stories are no longer accepted/rejected during the Sprint Review (since 2013). Accepting/Rejecting Stories is the job of the Product Owner and happens during the Sprint, whenever a story is ready.
\end{description}
\end{minipage}}
\newline\newline Text about the Sprint Review\newline


\fbox{\begin{minipage}{35em}
\begin{description}
   \item [Name] Sprint Retrospective
   \item [Duration] Max. 3h (2h is sensible)
   \item [Participants] Scrum Team \textbf{Only}
   \item [Responsible] Scrum Master
   \item [Input] Pro/Contra for each Scrum Team Member. Scrum Master prepares Topics.
   \item [Meeting Goal] Know where we are and how to continue.
   \item [Output] Action Items (tackle 2/3 items in the next sprint)(sensible things should not be on the sprint board)
\end{description}
\end{minipage}}
\newline\newline This meeting is team and process focused. Other meetings are focused on the product.\newline

\subsubsection{Defintion of \ldots}

Definiton of Ready:
\begin{itemize}
		\item Story has Business Value
    \item Story has been reviewed and estimated by the team
    \item Story is complete in format - User X needs to Y so they can Z
    \item Acceptance criteria are clear and agreed upon
		\item dependencies identified and ready or mitigation agreed
		\item Product Owner has approved the story
\end{itemize}

Definiton of Done:
\begin{itemize}
    \item Acceptance tests written and pass (Meets acceptance criteria)
    \item Unit and Integration tests written and pass
    \item Code peer reviewed and approved
		\item Code is meeting development standards (e.g. as defined in Checkstyle)
    \item Code committed to repository
		\item Deployed to system test environment and passes system tests
		\item Any build/deployment/configuration changes implemented/documented/communicated
    \item QA testing done (by team members other than those working on the implementation of that feature)
		\item Smoke tests performed on dev/staging environment
    \item Product Owner signed off
\end{itemize}

Release Definiton of Done:
\begin{itemize}
    \item Acceptance, Unit and Integration tests pass
    \item Configuration changes are ready
		\item Tag created for the release
\end{itemize}

Bug Definiton of Done:
\begin{itemize}
    \item The Defect is fixed
    \item if missing, Unit Tests are implemented that covers the fix’s code functionality
		\item a functional test to cover the bug is performed by a person other than the person who fixed the bug
		\item an automated test is created that checks the flow
\end{itemize}

\subsubsection{Stuff}
\enquote{the value of...}.
Some more lorem ipsum. Some more lorem ipsum. Examined her own past and wrote rather gloomily:
\blockquote{The past is not a peaceful landscape lying there behind me, 
a country in which I can stroll wherever I please, 
and will gradually show me all its secret hills and dates. 
As I was moving forward, so it was crumbling. 
Some more text to make it longer than three lines. 
Maybe it is long enough now.}

On our teams we use the pattern "Yesterday's Weather" to pull stories into a sprint. The average velocity for the past three sprints in automatically calculated in our Scrum tooling. The only discussion is sprint loading. Is everyone 100\% there for the sprint. If not, we reduce the load appropriately.

Stories need to be small. Commit to a "feature" for a sprint. This sounds like an epic. We removed committment from the Scrum Guide because of management dysfunction. The term is forecast.

I have consistently seen the best agile teams hit 15 function points per developer month vs. the average waterfall developers delivering 2 function points per month. Function points are an international standard independent of technology and teams while velocity in story points used in Scrum is team dependent. 

\blockquote{Like a synchronized bike team where the leader cleaves the air, a successful Scrum team follows tried-and-true principles to realize value. Agile doesn’t mean doing whatever you want; it means working together toward a common vision and producing a potentially shippable product increment at the end of every Sprint. Keeping with the bike team analogy, the guy leading the bike team  [the Scrum Master] is parting the resistance. And the faster they go, the more resistance there is… and the more necessary it is for the Scrum Master to block and tackle.	

If the Scrum Master gets tired, he or she needs to peel off and bring up the rear, while another person assumes the lead role, breaks the resistance, and pedals harder than anyone else.

No analogy is perfect and the Product Owners always ask where they are in this picture. They are in the van running beside the bikers encouraging them to peddle faster and telling them to take the left fork ahead!

-- Jeff Sutherland
}

\textbf{Do it before Lunch}
In "The Art Of Agile Planning" training course with Jim Shore and Diana Larsen, they recommend not doing standup meetings first thing in the morning.  The problem with early standups can be that people tend to wait around to start their day with that meeting.  And inevitably, people are late to work, which can screw up the whole thing. 

Their preferred time is a little bit before lunch.  That makes sure everyone is around.  Plus it gives people time to prepare in the morning so they won't just sit there trying to recall what they did the day before. 

The most important thing: it helps keep the meeting short.  Nobody wants to cut into lunch time!



